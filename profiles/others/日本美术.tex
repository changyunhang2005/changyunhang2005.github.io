\documentclass{ctexart}
\bibliographystyle{plain}
\begin{document}

\begin{titlepage}
   \begin{center}
    \vspace*{1cm}
    \Huge
    \textbf{浅析浮世绘风景画的特色与意蕴}
    \vspace{0.3cm}
    \vfill
    \huge
    \textbf{常运航}\\
    \vspace{0.8cm}
    \vspace{1.5cm}
    \LARGE      
    PB22000086\\
    少年班学院3班\\
    \today\\
   \end{center}
\end{titlepage}
\pagestyle{empty}
\begin{center}
    \huge
    \textbf{浅析浮世绘风景画的特色与意蕴}
    \\
    \vspace{0.4cm}
    \LARGE
    \vspace{0.4cm}
    \textbf{常运航}
\end{center}
【摘要】浮世绘是日本江户时代市民阶层创造出来的符合大众审美的绘画样式,主要有美人画、役者绘、风景画三大类。其中的风景画采用优秀的笔法展现了娟秀的山水风光,具有极高的艺术价值。本文将以浮世绘中的风景画为研究对象,结合特定作品,赏析浮世绘的艺术美感和审美特色,解析它的文化内涵和对后世的影响
\\
【关键字】浮世绘、风景画、特色
\vspace{1.5cm}
\section{浮世绘的介绍}
浮世绘在日本江户时代兴起,是主要描绘日本市井阶层和风景名胜的民族艺术。其中的“浮世”一词为佛教用语,指的是飘忽不定的世俗人间。“浮世绘”就是这样的一个描绘世俗生活和美景的一类艺术形式。早期的江户聚集的众多的武士,男女比例失调,因此江户的花柳业十分发达,从而早期的浮世绘主要以描绘花街女子的日常生活为主。此时的浮世绘多以笔墨绘制,随着市场对浮世绘的需求增大,浮世绘变转为由木板印制。多样的颜色和拓板技术的发展使浮世绘进入鼎盛时期。到了江户时代后期,风景画取代了美人画和役人画,成为了浮世绘的最后一束光芒。\cite{对日本浮世绘之探析}浮世绘体现了日本市民阶层的审美趣味,对后世的艺术影响深远,被誉为“江户时代的社会生活百科全书”。
\section{浮世绘风景画的特色}
日本娟秀的山水风光和优美的四季景色不仅培养了日本民族独特的民族审美,也为艺术家们提供了源源不尽的素材与灵感。浮世绘中的山水景色,体现的是市民阶层的自然审美:清雅与朴素。古代的画师凭借独特的绘画技巧,展现了日本人民心目中的自然风光。\cite{浮世绘审美特征研究}
\subsection{线条美}
线条是东方艺术的重要表现形式,线条曲折笔触勾勒的是艺术形象的韵律与气质。它的粗细疏密可以传达出作者的情感,激发出观者的审美感受。浮世绘主要采用版画绘制,画面的线条往往工整均衡,丰富饱满,对细节往往也有精致的勾勒,从而使得画面显得繁密但不失秩序。作品中的勾线多是黑色的实线,可以给人以清晰的观感。由于浮世绘版画的色彩缺少深浅的变化,作品中的线条就承担起了一部分写形和传神的任务。艺术家们用凝练柔美的笔画,塑造出平面化,意象化的艺术形象。
\subsection{色彩美}
色彩是绘画语言的重要组成部分,展现了绘画者乃至当时人民的色彩观和审美观。浮世绘的色彩使用也在一定程度上反映了江户时期日本人民的艺术情趣。浮世绘画师一般不追求在画面中完美复现出现实中的色彩,而是进行大胆的概括与简练,多使用纯度很高的色彩平涂在黑色线条之内,显现出饱和明净,素雅和谐的氛围。这样的色彩搭配也使得风景画具有装饰性和意象性的特点。明丽鲜艳的色彩往往能给人们带来强大的视觉冲击力,充分调动观者的感官和情感。如《富岳三十六景 凯风快晴》使用红色大面积平铺出富士山的山体,再用渐变的颜色渲染天空,配以白色和黑色的纹路,让人过目难忘。除此以外,风景画还大多使用典雅素静的色调来体现日本人民对素朴空灵的气质的追求。
\subsection{构图美}
浮世绘的构图一般不追求精准的透视和完整的纵深,浮世绘在形式上为平面式的画作,但也吸收了一部分中国画法和西方画法的优点。浮世绘画师吸收了南宋水墨画师马远,夏圭的“残山剩水”中大量留白的特点,用“不全之全”的构图美体现追求诗意的审美。同时画师也融合了西方绘画的透视技法,创造性的运用在了景物的整体效果上。画师通过对画面中的物体位置的巧妙安排,构造出和谐与均匀的画面风格。如《江户名所百景 龟户梅屋铺》中,粗壮的树干占据了中心的大量篇幅,观者好像就是在树干的后面观察着赏花的游客。这样的构图具有非常大的观赏魅力。
\section{浮世绘风景画的文化内涵}
\subsection{对自然的崇尚}
优美娟丽的本土风光塑造了日本民众对自然风光的欣赏和崇尚。他们将自然山川中的生灵当做具有生命感的神灵来崇拜。人与自然物的情感是共通的,他们将自己的情感融入大自然,用娟秀的笔触描绘着自然无穷的魅力。浮世绘风景画体现出对自然真挚的歌颂,展现了日本民众鲜活的自然观和审美观。
\subsection{对诗意的追求}
在浮世绘中,美人画总是体现着生机与明艳,但风景画却体现着诗意与清雅\cite{对日本浮世绘之探析}日本绘画师继承了道家天人合一的哲学理念,将光与影,明与暗,富有深度的禅意蕴含在其中。浮世绘风景画多选择具有诗意的意象,用娟秀的笔触将其勾勒出来,填上清新明丽的的色调,配以留白的构图,将景物的诗意展现的淋漓尽致。如歌川广重的《高轮之明月》描绘的成对的飞鸟,平静的湖面,淡淡的浮云,向我们构建了一个宁静致远的氛围,体现了作者的审美情趣
\subsection{对生命的探究}
浮世绘在描绘景物的同时也在探讨着独特的禅意与哲学。这是感性与理性的融合。“浮世”即现世,浮云般缥缈的现实包含着淡淡的哀伤,这是对生命本身的思考和对死亡本身的诘问。浮世若梦,万古不变的美好,唯有娟秀的山川。在风景画中,我们总是能够体验到画师对生命的敬畏,对死亡的思考。这其中反而饱含日本人民对生命的无限热爱和积极的人生态度。\cite{日本浮世绘版画的艺术美感及其文化内涵}
\section{浮世绘对日后艺术的影响 }
19世纪,随着浮世绘渐渐传入欧洲,这类独特的艺术风格吸引了一众艺术家的注意,其中尤以印象派画家为甚。年轻画家在极力寻找道路摆脱学院派传统思想的束缚时,浮世绘独特的线条,色彩,构图渲染出来的东方韵味成为了他们在创新路上的踏板。\cite{浮世魅影}印象派画家们钟情于收藏浮世绘,对其进行欣赏与借鉴。如大师莫奈的油画作品《穿日本和服的夫人》中,画面中的女子穿着传统和服,颇具有浮世绘的意蕴。梵高的作品《盛开的梅树》也是临摹《龟户田园》而来。浮世绘的东方韵味促使印象派画家投入了无限的大自然中寻找灵感,同时也让他们的线条勾勒和平面涂色展现了独特的东方气韵。
\section{总结}
浮世绘是极具东方特色的日本传统美术。它体现着江户时期日本市民阶层的独特审美,具有极高的艺术价值。浮世绘中的风景画更是凭借着它娟秀的线条,清丽的色彩,留白的构图,描绘了日本优美的山水风光,体现了东方独特的文化内涵。同时,浮世绘风景画也对后世的艺术产生了深远的影响,在艺术史中占有独特的地位。
\bibliography{ref}
\end{document}