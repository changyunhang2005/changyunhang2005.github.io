\documentclass{ctexart}
\usepackage{graphicx}
\bibliographystyle{plain}
\begin{document}

\begin{titlepage}
   \begin{center}
    \vspace*{1cm}
    \Huge
    \textbf{以房地产市场为背景的微观经济学概览}
    \vspace{0.3cm}
    \vfill
    \huge
    \textbf{常运航}\\
    \vspace{0.8cm}
    \vspace{1.5cm}
    \LARGE      
    PB22000086\\
    少年班学院3班\\
    \today\\
   \end{center}
\end{titlepage}
\pagestyle{empty}
\begin{center}
    \huge
    \textbf{以房地产市场为背景的微观经济学概览}
    \\
    \vspace{0.4cm}
    \LARGE
    \vspace{0.4cm}
    \textbf{常运航}
\end{center}

\vspace{1.5cm}
\section{前言}
从改革开放开始中央推行商品房制度到1998年全国商品房供给制度建立,房地产市场已经成为了中国经济的一大支柱,在市场经济中占据着重要地位。但与此同时,不合理的供给结构导致的市场矛盾也逐渐呈现,各专家学者也都就市场问题展开激烈讨论。而微观经济学作为从个体开始的对需求和供给展开研究的学科,自然能在房地产市场中找到许多活生生的例子。本文将以房地产市场为背景,回顾本学期微观经济学的内容,总结我对微观经济学的感悟与思考。

\section{市场中的供需}
\subsection{需求与供给}
商品房供给制度确立后,房地产就成为了由买房者和开发商组成的市场,以此为基础,商品房的定价就大体上由买房者的需求和开发商的供给决定。
\par
买房者的保留价格为愿意接受的商品房的最高价格,需求量即为对应价格愿意且能够购买的商品房数量,将此数量与商品价格之间的对应关系表现出来即为需求曲线。商品房价格,质量,经济状况等都能够影响需求量。
\par
开发商的供给是伴随着一系列商品房的不同价格而产生的一系列数量,我们同样可以用供给曲线表示两者的关系。影响商品房供给的主要有土地价格,边际机会成本,国家政策等。它们会使供给曲线发生不同程度的移动。
\par
生产者剩余是开发商在交易中的获利,消费者剩余是买房者保留价格和市场价格之间的差额。需求曲线和供给曲线相交可以得到市场均衡,此时的价格为均衡价格。我们可以通过马歇尔的局部均衡分析方法来考查各因素对市场价格的影响,如当其他条件不变时,买房者数量的变化会导致需求曲线的移动,从而改变均衡价格。
\subsection{选择与弹性}
弹性是定量预测需求和供给的工具,需求弹性度量了需求量对房价变化的反应程度,体现了房价升高时,消费者逃脱市场的能力大小。对具体的房地产问题,不同人群的需求弹性的大小不同,有住房刚需的人群弹性较小,炒房者需求弹性较大。供给弹性反应了供给量对价格变化的反应程度,主要由单位建造成本随产量增加而上升的速度,要素投入在投入要素市场中所占份额等因素决定。对房地产征税,就是使供给和需求之间产生一个楔子的过程。征税后,销量必定下降,买房者支付价格上升,开发商得到价格下降,税收的转嫁由需求和供给弹性决定。中国在实行房地产税改革时,一定要考虑税收的转嫁问题,保障消费者的利益。

\section{市场中的消费者和企业}
\subsection{消费者}
在微观经济学中,消费者通常假定为理性的,他们的行为总是遵循一定的规律。如边际效用递减规律决定了多套房产购入的效用随购买的每个单位递减,于是购房者通常选择将多套房源中的一部分出租。边际相等原则决定了消费者以最佳方案使用其收入,于是部分消费者会选择租房而不是购房。
\subsection{企业}
房地产市场能够满足完全竞争市场的一些特征:企业数量足够多,所有企业生产几乎同质的产品。开发商的生产即为将土地,原材料等投入转化为商品房的行动。完全竞争市场通过$P=MC$这一等式使生产总成本最小化。房地产的高额利润孤立企业进入,而亏损则促使企业退出,这体现了房地产市场的消除性原理。长期来看,房地产产业属于成本递减行业,单位房源的成本随产出的增加而下降。


\section{市场中的政府}
商品房诞生数十年来,房地产市场中的一些乱象逐渐暴露出来,部分城市的房价逐渐畸形,市场出现了一定程度上的失灵状况。面对失灵的市场,政府可以提出一定的措施来对市场进行管制。如2008年金融危机时,为稳定经济增长,防止房价大幅下跌,政府采取了一系列措施刺激房价。近些年,中央提出“房子是用来住的不是用来炒的”,用实际工作保障房价的稳定。
\section{总结}
住房需求时民众的一大重要需求,保障房地产市场稳定是一项重要的工作。经过微观经济学课程的学习,我们可以使用更加科学合理的视角和理论来分析房地产市场。同时,房地产市场给我们提供的大量现实实例可以帮助我们理解微观经济学中的概念与理论。
\end{document}