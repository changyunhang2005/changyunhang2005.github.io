\documentclass{ctexart}
\usepackage{graphicx}
\bibliographystyle{plain}
\begin{document}

\begin{titlepage}
   \begin{center}
    \vspace*{1cm}
    \Huge
    \textbf{买断制游戏平台市场考查}
    \vspace{0.3cm}
    \vfill
    \huge
    \textbf{常运航}\\
    \vspace{0.8cm}
    \vspace{1.5cm}
    \LARGE      
    PB22000086\\
    少年班学院3班\\
    \today\\
   \end{center}
\end{titlepage}
\pagestyle{empty}
\begin{center}
    \huge
    \textbf{买断制游戏平台市场考查}
    \\
    \vspace{0.4cm}
    \LARGE
    \vspace{0.4cm}
    \textbf{常运航}
\end{center}

\vspace{1.5cm}
\section{前言}
随着经济的不断发展,人民的娱乐需求不断发展。其中中国主机游戏市场在本世代(2017年后)迎来了快速发展,越来越多的人选择主机或PC作为娱乐手段。与更大众的手游与网游不同,买断制游戏市场虽然在中国是一个新兴的市场,但与传统的“一手交钱一手交货”的交易市场仍有着非常多的共性。一个完整的买断制游戏市场通常由游戏开发者(生产者),游戏平台(中间商),玩家(消费者)。游戏开发商要吸引玩家只需要注意游戏的制作与宣发即可,不同的游戏平台之间会如何进行市场竞争的呢?本文将对主机与PC上的各个平台进行考查,运用微观经济学的知识,探究买断制游戏平台市场竞争的现状。

\section{主机平台}
主机,也即游戏机。每家厂商制作的主机都会搭载自己独有的游戏平台,不同厂商的主机平台是不互通的,这就导致消费者需求曲线的弹性很小,因为用户更换主机厂商的成本较高。另一方面,主机厂商只承担主机的制造成本而一般不承担第三方游戏的开发成本,这就导致主机制造和平台供应几乎是一个成本不变行业。因此主机厂商最重要的是通过各种手段吸引新用户的加入并保证平台内游戏的数量和质量以此稳固旧用户的使用。本世代主流的游戏主机为索尼的PS5,任天堂的Switch,微软的Xbox,接下来将分别考查它们的市场竞争方式。
\subsection{索尼}
\subsubsection{优质的机器性能}
在主机“御三家”中,索尼的PS5可以做到在较低的价格下保持机器性能在中高端PC水平,较高的性价比使得它吸引了大量想要以较高画质体验游戏的玩家。
\subsubsection{限时的主机独占}
部分索尼旗下工作室的游戏和其他一些游戏会采取限时主机独占的策略,即在发售的一段时间内只登录PS5一家平台,从而达成部分游戏的销售垄断,部分玩家则不得不选择此平台。独占策略固然会影响消费者的体验,但却实打实得带来了主机销量的增长。
\subsection{任天堂}
\subsubsection{第一方工作室的完全独占}
任天堂除了制造主机以外也是游戏的开发商,与索尼一样,任天堂也有第一方游戏的独占策略,只不过这部分游戏仅仅只登录Switch平台。尽管此策略让玩家不满,但为了体验任天堂新的第一方游戏,玩家只能咬咬牙购买性能已经有些老旧的Switch。
\subsubsection{更好的便携性}
Switch是这三种主机中唯一一台掌机,独一无二的便携性可以开拓新的市场,从而在掌机领域达成一定程度上的垄断地位。但作为掌机领域的开拓者,我们可以认为这在一定程度上是有效的竞争。
\subsection{微软}
\subsubsection{不同的硬件配置}
微软为了扩大消费者数量,推出了两款硬件和价格档次都有所不同的主机xsx和xss,多样化的选择种类可以满足不同消费者的消费需求,从而扩大Xbox的玩家群体。
\subsubsection{订阅制的收费方式}
除了买断制之外,微软还提供了名为xgp的订阅制付费手段,每月仅需很少的钱就可以游玩游戏库中的大量游戏,这是一种类似于出租的消费方式,可以满足那些不想花大价钱也想体验大型游戏的玩家,从而增加Xbox的吸引力。

\section{PC平台}
PC的销售一般是各个电脑企业,因此PC端上的买断制游戏只需要建立一个销售的平台即可。一般的销售平台公司甚至不需要进行生产,只需要从营业额中抽成来维持平台的运行即可。和主机平台一样,PC端的不同平台游戏数据并不互通,因此玩家更换平台的成本依然较高,同时游戏开发商和玩家在选择平台是都有较大的抱团性和惯性,因此很容易像聊天软件平台一样形成一家独大的垄断局面。Steam平台依靠先发优势,稳定的游戏体验,定时的打折促销活动逐渐占据了绝大部分的市场份额,可以看到,即使Steam平台抽成达到$30\%$,游戏开发商和玩家大部分依然选择这个平台。但今年来又有Epic平台依靠不定期免费赠送游戏的方式吸引了一部分用户,但因为平台的使用体验一直不如原来的Steam,所以并没有能够撼动Steam的垄断地位
\section{总结}
本文依靠微观经济学的概念和方法,简要分析了买断制游戏平台的特点,并逐一分析了不同平台的销售与竞争策略,可以看到,各家平台都在为达到某一垄断地位而吸引玩家,而玩家一方面希望有单一的平台一劳永逸,一方面又不希望因为平台垄断而损失利益。希望各平台在进行商业竞争的同时,遵守市场秩序,不能损害消费者的利益。
\end{document}